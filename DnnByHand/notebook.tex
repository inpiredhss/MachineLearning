
% Default to the notebook output style

    


% Inherit from the specified cell style.




    
\documentclass[11pt]{article}

    
    
    \usepackage[T1]{fontenc}
    % Nicer default font (+ math font) than Computer Modern for most use cases
    \usepackage{mathpazo}

    % Basic figure setup, for now with no caption control since it's done
    % automatically by Pandoc (which extracts ![](path) syntax from Markdown).
    \usepackage{graphicx}
    % We will generate all images so they have a width \maxwidth. This means
    % that they will get their normal width if they fit onto the page, but
    % are scaled down if they would overflow the margins.
    \makeatletter
    \def\maxwidth{\ifdim\Gin@nat@width>\linewidth\linewidth
    \else\Gin@nat@width\fi}
    \makeatother
    \let\Oldincludegraphics\includegraphics
    % Set max figure width to be 80% of text width, for now hardcoded.
    \renewcommand{\includegraphics}[1]{\Oldincludegraphics[width=.8\maxwidth]{#1}}
    % Ensure that by default, figures have no caption (until we provide a
    % proper Figure object with a Caption API and a way to capture that
    % in the conversion process - todo).
    \usepackage{caption}
    \DeclareCaptionLabelFormat{nolabel}{}
    \captionsetup{labelformat=nolabel}

    \usepackage{adjustbox} % Used to constrain images to a maximum size 
    \usepackage{xcolor} % Allow colors to be defined
    \usepackage{enumerate} % Needed for markdown enumerations to work
    \usepackage{geometry} % Used to adjust the document margins
    \usepackage{amsmath} % Equations
    \usepackage{amssymb} % Equations
    \usepackage{textcomp} % defines textquotesingle
    % Hack from http://tex.stackexchange.com/a/47451/13684:
    \AtBeginDocument{%
        \def\PYZsq{\textquotesingle}% Upright quotes in Pygmentized code
    }
    \usepackage{upquote} % Upright quotes for verbatim code
    \usepackage{eurosym} % defines \euro
    \usepackage[mathletters]{ucs} % Extended unicode (utf-8) support
    \usepackage[utf8x]{inputenc} % Allow utf-8 characters in the tex document
    \usepackage{fancyvrb} % verbatim replacement that allows latex
    \usepackage{grffile} % extends the file name processing of package graphics 
                         % to support a larger range 
    % The hyperref package gives us a pdf with properly built
    % internal navigation ('pdf bookmarks' for the table of contents,
    % internal cross-reference links, web links for URLs, etc.)
    \usepackage{hyperref}
    \usepackage{longtable} % longtable support required by pandoc >1.10
    \usepackage{booktabs}  % table support for pandoc > 1.12.2
    \usepackage[inline]{enumitem} % IRkernel/repr support (it uses the enumerate* environment)
    \usepackage[normalem]{ulem} % ulem is needed to support strikethroughs (\sout)
                                % normalem makes italics be italics, not underlines
    

    
    
    % Colors for the hyperref package
    \definecolor{urlcolor}{rgb}{0,.145,.698}
    \definecolor{linkcolor}{rgb}{.71,0.21,0.01}
    \definecolor{citecolor}{rgb}{.12,.54,.11}

    % ANSI colors
    \definecolor{ansi-black}{HTML}{3E424D}
    \definecolor{ansi-black-intense}{HTML}{282C36}
    \definecolor{ansi-red}{HTML}{E75C58}
    \definecolor{ansi-red-intense}{HTML}{B22B31}
    \definecolor{ansi-green}{HTML}{00A250}
    \definecolor{ansi-green-intense}{HTML}{007427}
    \definecolor{ansi-yellow}{HTML}{DDB62B}
    \definecolor{ansi-yellow-intense}{HTML}{B27D12}
    \definecolor{ansi-blue}{HTML}{208FFB}
    \definecolor{ansi-blue-intense}{HTML}{0065CA}
    \definecolor{ansi-magenta}{HTML}{D160C4}
    \definecolor{ansi-magenta-intense}{HTML}{A03196}
    \definecolor{ansi-cyan}{HTML}{60C6C8}
    \definecolor{ansi-cyan-intense}{HTML}{258F8F}
    \definecolor{ansi-white}{HTML}{C5C1B4}
    \definecolor{ansi-white-intense}{HTML}{A1A6B2}

    % commands and environments needed by pandoc snippets
    % extracted from the output of `pandoc -s`
    \providecommand{\tightlist}{%
      \setlength{\itemsep}{0pt}\setlength{\parskip}{0pt}}
    \DefineVerbatimEnvironment{Highlighting}{Verbatim}{commandchars=\\\{\}}
    % Add ',fontsize=\small' for more characters per line
    \newenvironment{Shaded}{}{}
    \newcommand{\KeywordTok}[1]{\textcolor[rgb]{0.00,0.44,0.13}{\textbf{{#1}}}}
    \newcommand{\DataTypeTok}[1]{\textcolor[rgb]{0.56,0.13,0.00}{{#1}}}
    \newcommand{\DecValTok}[1]{\textcolor[rgb]{0.25,0.63,0.44}{{#1}}}
    \newcommand{\BaseNTok}[1]{\textcolor[rgb]{0.25,0.63,0.44}{{#1}}}
    \newcommand{\FloatTok}[1]{\textcolor[rgb]{0.25,0.63,0.44}{{#1}}}
    \newcommand{\CharTok}[1]{\textcolor[rgb]{0.25,0.44,0.63}{{#1}}}
    \newcommand{\StringTok}[1]{\textcolor[rgb]{0.25,0.44,0.63}{{#1}}}
    \newcommand{\CommentTok}[1]{\textcolor[rgb]{0.38,0.63,0.69}{\textit{{#1}}}}
    \newcommand{\OtherTok}[1]{\textcolor[rgb]{0.00,0.44,0.13}{{#1}}}
    \newcommand{\AlertTok}[1]{\textcolor[rgb]{1.00,0.00,0.00}{\textbf{{#1}}}}
    \newcommand{\FunctionTok}[1]{\textcolor[rgb]{0.02,0.16,0.49}{{#1}}}
    \newcommand{\RegionMarkerTok}[1]{{#1}}
    \newcommand{\ErrorTok}[1]{\textcolor[rgb]{1.00,0.00,0.00}{\textbf{{#1}}}}
    \newcommand{\NormalTok}[1]{{#1}}
    
    % Additional commands for more recent versions of Pandoc
    \newcommand{\ConstantTok}[1]{\textcolor[rgb]{0.53,0.00,0.00}{{#1}}}
    \newcommand{\SpecialCharTok}[1]{\textcolor[rgb]{0.25,0.44,0.63}{{#1}}}
    \newcommand{\VerbatimStringTok}[1]{\textcolor[rgb]{0.25,0.44,0.63}{{#1}}}
    \newcommand{\SpecialStringTok}[1]{\textcolor[rgb]{0.73,0.40,0.53}{{#1}}}
    \newcommand{\ImportTok}[1]{{#1}}
    \newcommand{\DocumentationTok}[1]{\textcolor[rgb]{0.73,0.13,0.13}{\textit{{#1}}}}
    \newcommand{\AnnotationTok}[1]{\textcolor[rgb]{0.38,0.63,0.69}{\textbf{\textit{{#1}}}}}
    \newcommand{\CommentVarTok}[1]{\textcolor[rgb]{0.38,0.63,0.69}{\textbf{\textit{{#1}}}}}
    \newcommand{\VariableTok}[1]{\textcolor[rgb]{0.10,0.09,0.49}{{#1}}}
    \newcommand{\ControlFlowTok}[1]{\textcolor[rgb]{0.00,0.44,0.13}{\textbf{{#1}}}}
    \newcommand{\OperatorTok}[1]{\textcolor[rgb]{0.40,0.40,0.40}{{#1}}}
    \newcommand{\BuiltInTok}[1]{{#1}}
    \newcommand{\ExtensionTok}[1]{{#1}}
    \newcommand{\PreprocessorTok}[1]{\textcolor[rgb]{0.74,0.48,0.00}{{#1}}}
    \newcommand{\AttributeTok}[1]{\textcolor[rgb]{0.49,0.56,0.16}{{#1}}}
    \newcommand{\InformationTok}[1]{\textcolor[rgb]{0.38,0.63,0.69}{\textbf{\textit{{#1}}}}}
    \newcommand{\WarningTok}[1]{\textcolor[rgb]{0.38,0.63,0.69}{\textbf{\textit{{#1}}}}}
    
    
    % Define a nice break command that doesn't care if a line doesn't already
    % exist.
    \def\br{\hspace*{\fill} \\* }
    % Math Jax compatability definitions
    \def\gt{>}
    \def\lt{<}
    % Document parameters
    \title{NetWithNumpy}
    
    
    

    % Pygments definitions
    
\makeatletter
\def\PY@reset{\let\PY@it=\relax \let\PY@bf=\relax%
    \let\PY@ul=\relax \let\PY@tc=\relax%
    \let\PY@bc=\relax \let\PY@ff=\relax}
\def\PY@tok#1{\csname PY@tok@#1\endcsname}
\def\PY@toks#1+{\ifx\relax#1\empty\else%
    \PY@tok{#1}\expandafter\PY@toks\fi}
\def\PY@do#1{\PY@bc{\PY@tc{\PY@ul{%
    \PY@it{\PY@bf{\PY@ff{#1}}}}}}}
\def\PY#1#2{\PY@reset\PY@toks#1+\relax+\PY@do{#2}}

\expandafter\def\csname PY@tok@w\endcsname{\def\PY@tc##1{\textcolor[rgb]{0.73,0.73,0.73}{##1}}}
\expandafter\def\csname PY@tok@c\endcsname{\let\PY@it=\textit\def\PY@tc##1{\textcolor[rgb]{0.25,0.50,0.50}{##1}}}
\expandafter\def\csname PY@tok@cp\endcsname{\def\PY@tc##1{\textcolor[rgb]{0.74,0.48,0.00}{##1}}}
\expandafter\def\csname PY@tok@k\endcsname{\let\PY@bf=\textbf\def\PY@tc##1{\textcolor[rgb]{0.00,0.50,0.00}{##1}}}
\expandafter\def\csname PY@tok@kp\endcsname{\def\PY@tc##1{\textcolor[rgb]{0.00,0.50,0.00}{##1}}}
\expandafter\def\csname PY@tok@kt\endcsname{\def\PY@tc##1{\textcolor[rgb]{0.69,0.00,0.25}{##1}}}
\expandafter\def\csname PY@tok@o\endcsname{\def\PY@tc##1{\textcolor[rgb]{0.40,0.40,0.40}{##1}}}
\expandafter\def\csname PY@tok@ow\endcsname{\let\PY@bf=\textbf\def\PY@tc##1{\textcolor[rgb]{0.67,0.13,1.00}{##1}}}
\expandafter\def\csname PY@tok@nb\endcsname{\def\PY@tc##1{\textcolor[rgb]{0.00,0.50,0.00}{##1}}}
\expandafter\def\csname PY@tok@nf\endcsname{\def\PY@tc##1{\textcolor[rgb]{0.00,0.00,1.00}{##1}}}
\expandafter\def\csname PY@tok@nc\endcsname{\let\PY@bf=\textbf\def\PY@tc##1{\textcolor[rgb]{0.00,0.00,1.00}{##1}}}
\expandafter\def\csname PY@tok@nn\endcsname{\let\PY@bf=\textbf\def\PY@tc##1{\textcolor[rgb]{0.00,0.00,1.00}{##1}}}
\expandafter\def\csname PY@tok@ne\endcsname{\let\PY@bf=\textbf\def\PY@tc##1{\textcolor[rgb]{0.82,0.25,0.23}{##1}}}
\expandafter\def\csname PY@tok@nv\endcsname{\def\PY@tc##1{\textcolor[rgb]{0.10,0.09,0.49}{##1}}}
\expandafter\def\csname PY@tok@no\endcsname{\def\PY@tc##1{\textcolor[rgb]{0.53,0.00,0.00}{##1}}}
\expandafter\def\csname PY@tok@nl\endcsname{\def\PY@tc##1{\textcolor[rgb]{0.63,0.63,0.00}{##1}}}
\expandafter\def\csname PY@tok@ni\endcsname{\let\PY@bf=\textbf\def\PY@tc##1{\textcolor[rgb]{0.60,0.60,0.60}{##1}}}
\expandafter\def\csname PY@tok@na\endcsname{\def\PY@tc##1{\textcolor[rgb]{0.49,0.56,0.16}{##1}}}
\expandafter\def\csname PY@tok@nt\endcsname{\let\PY@bf=\textbf\def\PY@tc##1{\textcolor[rgb]{0.00,0.50,0.00}{##1}}}
\expandafter\def\csname PY@tok@nd\endcsname{\def\PY@tc##1{\textcolor[rgb]{0.67,0.13,1.00}{##1}}}
\expandafter\def\csname PY@tok@s\endcsname{\def\PY@tc##1{\textcolor[rgb]{0.73,0.13,0.13}{##1}}}
\expandafter\def\csname PY@tok@sd\endcsname{\let\PY@it=\textit\def\PY@tc##1{\textcolor[rgb]{0.73,0.13,0.13}{##1}}}
\expandafter\def\csname PY@tok@si\endcsname{\let\PY@bf=\textbf\def\PY@tc##1{\textcolor[rgb]{0.73,0.40,0.53}{##1}}}
\expandafter\def\csname PY@tok@se\endcsname{\let\PY@bf=\textbf\def\PY@tc##1{\textcolor[rgb]{0.73,0.40,0.13}{##1}}}
\expandafter\def\csname PY@tok@sr\endcsname{\def\PY@tc##1{\textcolor[rgb]{0.73,0.40,0.53}{##1}}}
\expandafter\def\csname PY@tok@ss\endcsname{\def\PY@tc##1{\textcolor[rgb]{0.10,0.09,0.49}{##1}}}
\expandafter\def\csname PY@tok@sx\endcsname{\def\PY@tc##1{\textcolor[rgb]{0.00,0.50,0.00}{##1}}}
\expandafter\def\csname PY@tok@m\endcsname{\def\PY@tc##1{\textcolor[rgb]{0.40,0.40,0.40}{##1}}}
\expandafter\def\csname PY@tok@gh\endcsname{\let\PY@bf=\textbf\def\PY@tc##1{\textcolor[rgb]{0.00,0.00,0.50}{##1}}}
\expandafter\def\csname PY@tok@gu\endcsname{\let\PY@bf=\textbf\def\PY@tc##1{\textcolor[rgb]{0.50,0.00,0.50}{##1}}}
\expandafter\def\csname PY@tok@gd\endcsname{\def\PY@tc##1{\textcolor[rgb]{0.63,0.00,0.00}{##1}}}
\expandafter\def\csname PY@tok@gi\endcsname{\def\PY@tc##1{\textcolor[rgb]{0.00,0.63,0.00}{##1}}}
\expandafter\def\csname PY@tok@gr\endcsname{\def\PY@tc##1{\textcolor[rgb]{1.00,0.00,0.00}{##1}}}
\expandafter\def\csname PY@tok@ge\endcsname{\let\PY@it=\textit}
\expandafter\def\csname PY@tok@gs\endcsname{\let\PY@bf=\textbf}
\expandafter\def\csname PY@tok@gp\endcsname{\let\PY@bf=\textbf\def\PY@tc##1{\textcolor[rgb]{0.00,0.00,0.50}{##1}}}
\expandafter\def\csname PY@tok@go\endcsname{\def\PY@tc##1{\textcolor[rgb]{0.53,0.53,0.53}{##1}}}
\expandafter\def\csname PY@tok@gt\endcsname{\def\PY@tc##1{\textcolor[rgb]{0.00,0.27,0.87}{##1}}}
\expandafter\def\csname PY@tok@err\endcsname{\def\PY@bc##1{\setlength{\fboxsep}{0pt}\fcolorbox[rgb]{1.00,0.00,0.00}{1,1,1}{\strut ##1}}}
\expandafter\def\csname PY@tok@kc\endcsname{\let\PY@bf=\textbf\def\PY@tc##1{\textcolor[rgb]{0.00,0.50,0.00}{##1}}}
\expandafter\def\csname PY@tok@kd\endcsname{\let\PY@bf=\textbf\def\PY@tc##1{\textcolor[rgb]{0.00,0.50,0.00}{##1}}}
\expandafter\def\csname PY@tok@kn\endcsname{\let\PY@bf=\textbf\def\PY@tc##1{\textcolor[rgb]{0.00,0.50,0.00}{##1}}}
\expandafter\def\csname PY@tok@kr\endcsname{\let\PY@bf=\textbf\def\PY@tc##1{\textcolor[rgb]{0.00,0.50,0.00}{##1}}}
\expandafter\def\csname PY@tok@bp\endcsname{\def\PY@tc##1{\textcolor[rgb]{0.00,0.50,0.00}{##1}}}
\expandafter\def\csname PY@tok@fm\endcsname{\def\PY@tc##1{\textcolor[rgb]{0.00,0.00,1.00}{##1}}}
\expandafter\def\csname PY@tok@vc\endcsname{\def\PY@tc##1{\textcolor[rgb]{0.10,0.09,0.49}{##1}}}
\expandafter\def\csname PY@tok@vg\endcsname{\def\PY@tc##1{\textcolor[rgb]{0.10,0.09,0.49}{##1}}}
\expandafter\def\csname PY@tok@vi\endcsname{\def\PY@tc##1{\textcolor[rgb]{0.10,0.09,0.49}{##1}}}
\expandafter\def\csname PY@tok@vm\endcsname{\def\PY@tc##1{\textcolor[rgb]{0.10,0.09,0.49}{##1}}}
\expandafter\def\csname PY@tok@sa\endcsname{\def\PY@tc##1{\textcolor[rgb]{0.73,0.13,0.13}{##1}}}
\expandafter\def\csname PY@tok@sb\endcsname{\def\PY@tc##1{\textcolor[rgb]{0.73,0.13,0.13}{##1}}}
\expandafter\def\csname PY@tok@sc\endcsname{\def\PY@tc##1{\textcolor[rgb]{0.73,0.13,0.13}{##1}}}
\expandafter\def\csname PY@tok@dl\endcsname{\def\PY@tc##1{\textcolor[rgb]{0.73,0.13,0.13}{##1}}}
\expandafter\def\csname PY@tok@s2\endcsname{\def\PY@tc##1{\textcolor[rgb]{0.73,0.13,0.13}{##1}}}
\expandafter\def\csname PY@tok@sh\endcsname{\def\PY@tc##1{\textcolor[rgb]{0.73,0.13,0.13}{##1}}}
\expandafter\def\csname PY@tok@s1\endcsname{\def\PY@tc##1{\textcolor[rgb]{0.73,0.13,0.13}{##1}}}
\expandafter\def\csname PY@tok@mb\endcsname{\def\PY@tc##1{\textcolor[rgb]{0.40,0.40,0.40}{##1}}}
\expandafter\def\csname PY@tok@mf\endcsname{\def\PY@tc##1{\textcolor[rgb]{0.40,0.40,0.40}{##1}}}
\expandafter\def\csname PY@tok@mh\endcsname{\def\PY@tc##1{\textcolor[rgb]{0.40,0.40,0.40}{##1}}}
\expandafter\def\csname PY@tok@mi\endcsname{\def\PY@tc##1{\textcolor[rgb]{0.40,0.40,0.40}{##1}}}
\expandafter\def\csname PY@tok@il\endcsname{\def\PY@tc##1{\textcolor[rgb]{0.40,0.40,0.40}{##1}}}
\expandafter\def\csname PY@tok@mo\endcsname{\def\PY@tc##1{\textcolor[rgb]{0.40,0.40,0.40}{##1}}}
\expandafter\def\csname PY@tok@ch\endcsname{\let\PY@it=\textit\def\PY@tc##1{\textcolor[rgb]{0.25,0.50,0.50}{##1}}}
\expandafter\def\csname PY@tok@cm\endcsname{\let\PY@it=\textit\def\PY@tc##1{\textcolor[rgb]{0.25,0.50,0.50}{##1}}}
\expandafter\def\csname PY@tok@cpf\endcsname{\let\PY@it=\textit\def\PY@tc##1{\textcolor[rgb]{0.25,0.50,0.50}{##1}}}
\expandafter\def\csname PY@tok@c1\endcsname{\let\PY@it=\textit\def\PY@tc##1{\textcolor[rgb]{0.25,0.50,0.50}{##1}}}
\expandafter\def\csname PY@tok@cs\endcsname{\let\PY@it=\textit\def\PY@tc##1{\textcolor[rgb]{0.25,0.50,0.50}{##1}}}

\def\PYZbs{\char`\\}
\def\PYZus{\char`\_}
\def\PYZob{\char`\{}
\def\PYZcb{\char`\}}
\def\PYZca{\char`\^}
\def\PYZam{\char`\&}
\def\PYZlt{\char`\<}
\def\PYZgt{\char`\>}
\def\PYZsh{\char`\#}
\def\PYZpc{\char`\%}
\def\PYZdl{\char`\$}
\def\PYZhy{\char`\-}
\def\PYZsq{\char`\'}
\def\PYZdq{\char`\"}
\def\PYZti{\char`\~}
% for compatibility with earlier versions
\def\PYZat{@}
\def\PYZlb{[}
\def\PYZrb{]}
\makeatother


    % Exact colors from NB
    \definecolor{incolor}{rgb}{0.0, 0.0, 0.5}
    \definecolor{outcolor}{rgb}{0.545, 0.0, 0.0}



    
    % Prevent overflowing lines due to hard-to-break entities
    \sloppy 
    % Setup hyperref package
    \hypersetup{
      breaklinks=true,  % so long urls are correctly broken across lines
      colorlinks=true,
      urlcolor=urlcolor,
      linkcolor=linkcolor,
      citecolor=citecolor,
      }
    % Slightly bigger margins than the latex defaults
    
    \geometry{verbose,tmargin=1in,bmargin=1in,lmargin=1in,rmargin=1in}
    
    

    \begin{document}
    
    
    \maketitle
    
    

    
    \subparagraph{思考}\label{ux601dux8003}

\begin{itemize}
\tightlist
\item
  从零搭建一个神经网络模型
\item
  从感知机开始写起,一步步捋清神经网络的结构
\item
  激活函数怎么写、采用何种损失函数
\item
  前向传播怎么写、后向传播又怎么写,权值如何迭代更新
\end{itemize}

    \subparagraph{理论}\label{ux7406ux8bba}

\begin{itemize}
\tightlist
\item
  感知机模型、神经网络
\item
  神经网络中的输入层、隐藏层、输出层、权重与偏置
\item
  激活函数、损失函数、前向传播、反向传播、
\item
  权值更新、梯度下降、微积分中的链式求导、方向梯度
\end{itemize}

    \subparagraph{方法}\label{ux65b9ux6cd5}

\begin{itemize}
\tightlist
\item
  定义网络结构(指定输出层、隐藏层、输出层的大小)
\item
  初始化模型参数
\item
  循环操作:执行前向传播/计算损失/执行后向传播/权值更新
\end{itemize}

    \begin{Verbatim}[commandchars=\\\{\}]
{\color{incolor}In [{\color{incolor}3}]:} \PY{c+c1}{\PYZsh{} 定义激活函数}
        \PY{k+kn}{import} \PY{n+nn}{numpy} \PY{k}{as} \PY{n+nn}{np}
        \PY{k}{def} \PY{n+nf}{sigmod}\PY{p}{(}\PY{n}{x}\PY{p}{)}\PY{p}{:}
            \PY{k}{return} \PY{l+m+mi}{1}\PY{o}{/}\PY{p}{(}\PY{l+m+mi}{1}\PY{o}{+}\PY{n}{np}\PY{o}{.}\PY{n}{exp}\PY{p}{(}\PY{o}{\PYZhy{}}\PY{n}{x}\PY{p}{)}\PY{p}{)}
        
        \PY{n}{sigmod}\PY{p}{(}\PY{l+m+mi}{3}\PY{p}{)}
\end{Verbatim}


\begin{Verbatim}[commandchars=\\\{\}]
{\color{outcolor}Out[{\color{outcolor}3}]:} 0.9525741268224334
\end{Verbatim}
            
    \begin{Verbatim}[commandchars=\\\{\}]
{\color{incolor}In [{\color{incolor}18}]:} \PY{c+c1}{\PYZsh{} 模型参数初始化\PYZhy{}权值、偏置\PYZhy{}w,b}
         
         \PY{k}{def} \PY{n+nf}{initilize\PYZus{}with\PYZus{}zeros}\PY{p}{(}\PY{n}{dim}\PY{p}{)}\PY{p}{:}
             \PY{n}{w}\PY{o}{=}\PY{n}{np}\PY{o}{.}\PY{n}{zeros}\PY{p}{(}\PY{p}{(}\PY{n}{dim}\PY{p}{,}\PY{l+m+mi}{1}\PY{p}{)}\PY{p}{)}
             \PY{n}{b}\PY{o}{=}\PY{l+m+mf}{4.0}
         \PY{c+c1}{\PYZsh{}     assert(w.shape==(dim,1))}
         \PY{c+c1}{\PYZsh{}     assert(isinstance(b,float) or isinstance(b,int))}
             \PY{k}{return} \PY{n}{w}\PY{p}{,}\PY{n}{b}
         \PY{n}{initilize\PYZus{}with\PYZus{}zeros}\PY{p}{(}\PY{l+m+mi}{3}\PY{p}{)}   
\end{Verbatim}


\begin{Verbatim}[commandchars=\\\{\}]
{\color{outcolor}Out[{\color{outcolor}18}]:} (array([[0.],
                 [0.],
                 [0.]]), 4.0)
\end{Verbatim}
            
    \begin{itemize}
\tightlist
\item
  模型的主体部分
\item
  前向传播和计算损失、反向传播和权值更新
\item
  也是神经网络训练过程中每一次需要迭代的部分
\item
  前向传播简单而言就是计算预测 y 的过程
\item
  后向传播则是根据预测值和实际值之间的误差不断往回推更新权值和偏置的过程
\end{itemize}

    \begin{figure}
\centering
\includegraphics{attachment:image.png}
\caption{image.png}
\end{figure}

    \begin{itemize}
\tightlist
\item
  定义一个大的前向传播函数
\item
  预测值y为模型从输入到经过激活函数处理后的输出的结果
\item
  损失函数我们采用交叉熵损失
\end{itemize}

    \begin{Verbatim}[commandchars=\\\{\}]
{\color{incolor}In [{\color{incolor}61}]:} \PY{c+c1}{\PYZsh{} 前向传播的编写思路:}
         \PY{c+c1}{\PYZsh{} 激活函数直接表示感知机输出的预测值}
         \PY{c+c1}{\PYZsh{} 定义交叉熵损失函数计算了损失}
         \PY{c+c1}{\PYZsh{} 损失函数计算了权值 w 和偏置 b的梯度}
         \PY{c+c1}{\PYZsh{} 参数梯度结果以字典和损失作为函数输出返回}
         
         \PY{c+c1}{\PYZsh{} 从数组的形状中删除单维度条目,即把shape中为1的维度去掉}
         \PY{c+c1}{\PYZsh{} 用法:numpy.squeeze(a,axis = None)}
         
         \PY{k}{def} \PY{n+nf}{propagate}\PY{p}{(}\PY{n}{w}\PY{p}{,}\PY{n}{b}\PY{p}{,}\PY{n}{X}\PY{p}{,}\PY{n}{Y}\PY{p}{)}\PY{p}{:}
             \PY{n}{m}\PY{o}{=}\PY{n}{X}\PY{o}{.}\PY{n}{shape}\PY{p}{[}\PY{l+m+mi}{1}\PY{p}{]}
             \PY{n}{y}\PY{o}{=}\PY{n}{sigmoid}\PY{p}{(}\PY{n}{np}\PY{o}{.}\PY{n}{dot}\PY{p}{(}\PY{n}{w}\PY{o}{.}\PY{n}{T}\PY{p}{,}\PY{n}{X}\PY{p}{)}\PY{o}{+}\PY{n}{b}\PY{p}{)}
             \PY{n}{cost}\PY{o}{=}\PY{o}{\PYZhy{}}\PY{l+m+mi}{1}\PY{o}{/}\PY{n}{m}\PY{o}{*}\PY{n}{np}\PY{o}{.}\PY{n}{sum}\PY{p}{(}\PY{n}{Y}\PY{o}{*}\PY{n}{np}\PY{o}{.}\PY{n}{log}\PY{p}{(}\PY{n}{y}\PY{p}{)}\PY{o}{+}\PY{p}{(}\PY{l+m+mi}{1}\PY{o}{\PYZhy{}}\PY{n}{Y}\PY{p}{)}\PY{o}{*}\PY{n}{np}\PY{o}{.}\PY{n}{log}\PY{p}{(}\PY{l+m+mi}{1}\PY{o}{\PYZhy{}}\PY{n}{y}\PY{p}{)}\PY{p}{)}
             
             \PY{n}{dw}\PY{o}{=}\PY{n}{np}\PY{o}{.}\PY{n}{dot}\PY{p}{(}\PY{n}{X}\PY{p}{,}\PY{p}{(}\PY{n}{y}\PY{o}{\PYZhy{}}\PY{n}{Y}\PY{p}{)}\PY{o}{.}\PY{n}{T}\PY{p}{)}\PY{o}{/}\PY{n}{m}
             \PY{n}{db}\PY{o}{=}\PY{n}{np}\PY{o}{.}\PY{n}{sum}\PY{p}{(}\PY{n}{y}\PY{o}{\PYZhy{}}\PY{n}{Y}\PY{p}{)}\PY{o}{/}\PY{n}{m}
             \PY{k}{assert}\PY{p}{(}\PY{n}{dw}\PY{o}{.}\PY{n}{shape}\PY{o}{==}\PY{n}{w}\PY{o}{.}\PY{n}{shape}\PY{p}{)}
             \PY{k}{assert}\PY{p}{(}\PY{n}{db}\PY{o}{.}\PY{n}{dtype}\PY{o}{==}\PY{n+nb}{float}\PY{p}{)}
             
             \PY{n}{cost}\PY{o}{=}\PY{n}{np}\PY{o}{.}\PY{n}{squeeze}\PY{p}{(}\PY{n}{cost}\PY{p}{)}
             \PY{k}{assert}\PY{p}{(}\PY{n}{cost}\PY{o}{.}\PY{n}{shape}\PY{o}{==}\PY{p}{(}\PY{p}{)}\PY{p}{)}
             
             \PY{n}{grads}\PY{o}{=}\PY{p}{\PYZob{}}\PY{l+s+s1}{\PYZsq{}}\PY{l+s+s1}{dw}\PY{l+s+s1}{\PYZsq{}}\PY{p}{:}\PY{n}{dw}\PY{p}{,}
                   \PY{l+s+s1}{\PYZsq{}}\PY{l+s+s1}{db}\PY{l+s+s1}{\PYZsq{}}\PY{p}{:}\PY{n}{db}\PY{p}{\PYZcb{}}
             
             \PY{k}{return} \PY{n}{grads}\PY{p}{,} \PY{n}{cost}
\end{Verbatim}


    \begin{figure}
\centering
\includegraphics{attachment:image.png}
\caption{image.png}
\end{figure}

    \begin{figure}
\centering
\includegraphics{attachment:image.png}
\caption{image.png}
\end{figure}

    \begin{figure}
\centering
\includegraphics{attachment:image.png}
\caption{image.png}
\end{figure}

    \begin{figure}
\centering
\includegraphics{attachment:image.png}
\caption{image.png}
\end{figure}

    \begin{figure}
\centering
\includegraphics{attachment:image.png}
\caption{image.png}
\end{figure}

    \begin{itemize}
\tightlist
\item
  模型学习速度:学习率; 偏导值;
\item
  偏导值的大小取决于样本、模型错误程度;
\item
  逻辑函数得到概率,交叉熵作损失函数时,模型效果差学习速度较快
\item
  sigmoid(softmax)+cross-entropy loss 擅长于学习类间的信息
\item
  采用了类间竞争机制,只关心正确标签预测概率的准确性,忽略其他非正确标签的差异,导致学习到的特征比较散。优化:对softmax进行改进,如L-Softmax、SM-Softmax、AM-Softmax等
\end{itemize}

    \begin{figure}
\centering
\includegraphics{attachment:image.png}
\caption{image.png}
\end{figure}

    \begin{figure}
\centering
\includegraphics{attachment:image.png}
\caption{image.png}
\end{figure}

    \begin{figure}
\centering
\includegraphics{attachment:image.png}
\caption{image.png}
\end{figure}

    \begin{figure}
\centering
\includegraphics{attachment:image.png}
\caption{image.png}
\end{figure}

    \begin{Verbatim}[commandchars=\\\{\}]
{\color{incolor}In [{\color{incolor}49}]:} \PY{k+kn}{import} \PY{n+nn}{numpy} \PY{k}{as} \PY{n+nn}{np}
         \PY{c+c1}{\PYZsh{} x = np.array([[[0,1], [1,2], [2,3]],[[0,1], [1,2], [2,3]]])}
         \PY{n}{x} \PY{o}{=} \PY{n}{np}\PY{o}{.}\PY{n}{array}\PY{p}{(}\PY{p}{[}\PY{p}{[}\PY{p}{[}\PY{l+m+mi}{0}\PY{p}{,}\PY{l+m+mi}{1}\PY{p}{]}\PY{p}{,} \PY{p}{[}\PY{l+m+mi}{1}\PY{p}{,}\PY{l+m+mi}{2}\PY{p}{]}\PY{p}{,} \PY{p}{[}\PY{l+m+mi}{2}\PY{p}{,}\PY{l+m+mi}{3}\PY{p}{]}\PY{p}{]}\PY{p}{]}\PY{p}{)}
         \PY{n+nb}{print}\PY{p}{(}\PY{n}{x}\PY{p}{)}
          
         \PY{n+nb}{print}\PY{p}{(}\PY{n}{x}\PY{o}{.}\PY{n}{shape}\PY{p}{)}
         \PY{n+nb}{print}\PY{p}{(}\PY{n}{x}\PY{o}{.}\PY{n}{shape}\PY{p}{[}\PY{l+m+mi}{1}\PY{p}{]}\PY{p}{)}
         
         \PY{n}{d} \PY{o}{=} \PY{n}{np}\PY{o}{.}\PY{n}{squeeze}\PY{p}{(}\PY{n}{x}\PY{p}{)}   \PY{c+c1}{\PYZsh{} 从数组的形状中删除单维条目,即把shape中为1的维度去掉}
         \PY{n+nb}{print}\PY{p}{(}\PY{n}{d}\PY{o}{.}\PY{n}{shape}\PY{p}{)}
\end{Verbatim}


    \begin{Verbatim}[commandchars=\\\{\}]
[[[0 1]
  [1 2]
  [2 3]]]
(1, 3, 2)
3
(3, 2)

    \end{Verbatim}

    \begin{itemize}
\item
  backward\_propagation
\item
  计算每一步的当前损失 根据损失对权值进行更新
\item
  先是建立了一个损失列表容器
\item
  将前一步定义的前向传播函数放进去执行迭代操作
\item
  计算每一步的当前损失和梯度
\item
  利用梯度下降法对权值进行更新
\item
  用字典封装迭代结束时的参数和梯度进行返回
\end{itemize}

    \begin{Verbatim}[commandchars=\\\{\}]
{\color{incolor}In [{\color{incolor}52}]:} \PY{k}{def} \PY{n+nf}{backward\PYZus{}propagation}\PY{p}{(}\PY{n}{w}\PY{p}{,}\PY{n}{b}\PY{p}{,}\PY{n}{X}\PY{p}{,}\PY{n}{Y}\PY{p}{,}\PY{n}{num\PYZus{}iterations}\PY{p}{,}\PY{n}{learning\PYZus{}rate}\PY{p}{,}\PY{n}{print\PYZus{}cost}\PY{o}{=}\PY{k+kc}{False}\PY{p}{)}\PY{p}{:}
             \PY{n}{cost}\PY{o}{=}\PY{p}{[}\PY{p}{]}
             \PY{k}{for} \PY{n}{i} \PY{o+ow}{in} \PY{n+nb}{range}\PY{p}{(}\PY{n}{num\PYZus{}iterations}\PY{p}{)}\PY{p}{:}
                 \PY{n}{grad}\PY{p}{,}\PY{n}{cost}\PY{o}{=}\PY{n}{propagate}\PY{p}{(}\PY{n}{w}\PY{p}{,}\PY{n}{b}\PY{p}{,}\PY{n}{X}\PY{p}{,}\PY{n}{Y}\PY{p}{)}
                 
                 \PY{n}{dw}\PY{o}{=}\PY{n}{grad}\PY{p}{[}\PY{l+s+s1}{\PYZsq{}}\PY{l+s+s1}{dw}\PY{l+s+s1}{\PYZsq{}}\PY{p}{]}
                 \PY{n}{db}\PY{o}{=}\PY{n}{grad}\PY{p}{[}\PY{l+s+s1}{\PYZsq{}}\PY{l+s+s1}{db}\PY{l+s+s1}{\PYZsq{}}\PY{p}{]}
                 
                 \PY{n}{w}\PY{o}{=}\PY{n}{w}\PY{o}{\PYZhy{}}\PY{n}{learning\PYZus{}rate}\PY{o}{*}\PY{n}{dw}
                 \PY{n}{b}\PY{o}{=}\PY{n}{b}\PY{o}{\PYZhy{}}\PY{n}{learning\PYZus{}rate}\PY{o}{*}\PY{n}{db}
                 \PY{k}{if} \PY{n}{i}\PY{o}{\PYZpc{}}\PY{k}{100}==0:
                     \PY{n}{cost}\PY{o}{.}\PY{n}{append}\PY{p}{(}\PY{n}{cost}\PY{p}{)}
                 \PY{k}{if} \PY{n}{print\PYZus{}cost} \PY{o+ow}{and} \PY{n}{i}\PY{o}{\PYZpc{}}\PY{k}{100}==0:
                     \PY{n+nb}{print}\PY{p}{(}\PY{l+s+s2}{\PYZdq{}}\PY{l+s+s2}{cost after iteration }\PY{l+s+si}{\PYZpc{}i}\PY{l+s+s2}{: }\PY{l+s+si}{\PYZpc{}f}\PY{l+s+s2}{\PYZdq{}} \PY{o}{\PYZpc{}}\PY{p}{(}\PY{n}{i}\PY{p}{,}\PY{n}{cost}\PY{p}{)}\PY{p}{)}
             \PY{n}{params}\PY{o}{=}\PY{p}{\PYZob{}}\PY{l+s+s2}{\PYZdq{}}\PY{l+s+s2}{w}\PY{l+s+s2}{\PYZdq{}}\PY{p}{:}\PY{n}{w}\PY{p}{,}\PY{l+s+s2}{\PYZdq{}}\PY{l+s+s2}{b}\PY{l+s+s2}{\PYZdq{}}\PY{p}{:}\PY{n}{b}\PY{p}{\PYZcb{}}
             \PY{n}{grads}\PY{o}{=}\PY{p}{\PYZob{}}\PY{l+s+s2}{\PYZdq{}}\PY{l+s+s2}{dw}\PY{l+s+s2}{\PYZdq{}}\PY{p}{:}\PY{n}{dw}\PY{p}{,}\PY{l+s+s2}{\PYZdq{}}\PY{l+s+s2}{db}\PY{l+s+s2}{\PYZdq{}}\PY{p}{:}\PY{n}{db}\PY{p}{\PYZcb{}}
             \PY{k}{return} \PY{n}{params}\PY{p}{,}\PY{n}{grads}\PY{p}{,}\PY{n}{costs}
             
\end{Verbatim}


    \begin{itemize}
\tightlist
\item
  预测函数 predict,将模型的概率输出转化为0/1值
\end{itemize}

    \begin{Verbatim}[commandchars=\\\{\}]
{\color{incolor}In [{\color{incolor}55}]:} \PY{k}{def} \PY{n+nf}{predict}\PY{p}{(}\PY{n}{w}\PY{p}{,}\PY{n}{b}\PY{p}{,}\PY{n}{X}\PY{p}{)}\PY{p}{:}
             \PY{n}{m}\PY{o}{=}\PY{n}{X}\PY{o}{.}\PY{n}{shape}\PY{p}{[}\PY{l+m+mi}{1}\PY{p}{]}
             \PY{n}{Y\PYZus{}prediction}\PY{o}{=}\PY{n}{np}\PY{o}{.}\PY{n}{zeros}\PY{p}{(}\PY{p}{(}\PY{l+m+mi}{1}\PY{p}{,}\PY{n}{m}\PY{p}{)}\PY{p}{)}
             \PY{n}{w}\PY{o}{=}\PY{n}{w}\PY{o}{.}\PY{n}{reshape}\PY{p}{(}\PY{n}{X}\PY{o}{.}\PY{n}{shape}\PY{p}{[}\PY{l+m+mi}{0}\PY{p}{]}\PY{p}{,}\PY{l+m+mi}{1}\PY{p}{)}
             \PY{n}{A}\PY{o}{=}\PY{n}{sigmoid}\PY{p}{(}\PY{n}{np}\PY{o}{.}\PY{n}{dot}\PY{p}{(}\PY{n}{w}\PY{o}{.}\PY{n}{T}\PY{p}{,}\PY{n}{X}\PY{p}{)}\PY{o}{+}\PY{n}{b}\PY{p}{)}
             \PY{k}{for} \PY{n}{i} \PY{o+ow}{in} \PY{n+nb}{range}\PY{p}{(}\PY{n}{A}\PY{o}{.}\PY{n}{shape}\PY{p}{[}\PY{l+m+mi}{1}\PY{p}{]}\PY{p}{)}\PY{p}{:}
                 \PY{k}{if} \PY{n}{A}\PY{p}{[}\PY{p}{:}\PY{n}{i}\PY{p}{]}\PY{o}{\PYZgt{}}\PY{l+m+mf}{0.5}\PY{p}{:}
                     \PY{n}{Y\PYZus{}prediction}\PY{p}{[}\PY{p}{:}\PY{p}{,}\PY{n}{i}\PY{p}{]}\PY{o}{=}\PY{l+m+mi}{1}
                 \PY{k}{else}\PY{p}{:}
                     \PY{n}{Y\PYZus{}prediction}\PY{p}{[}\PY{p}{:}\PY{p}{,}\PY{n}{i}\PY{p}{]}\PY{o}{=}\PY{l+m+mi}{0}
             \PY{k}{assert}\PY{p}{(}\PY{n}{Y\PYZus{}prediction}\PY{o}{.}\PY{n}{shape}\PY{o}{==}\PY{p}{(}\PY{l+m+mi}{1}\PY{p}{,}\PY{n}{m}\PY{p}{)}\PY{p}{)}
             \PY{k}{return} \PY{n}{Y\PYZus{}prediction}
\end{Verbatim}


    \begin{itemize}
\tightlist
\item
  函数封装
\end{itemize}

    \begin{Verbatim}[commandchars=\\\{\}]
{\color{incolor}In [{\color{incolor}64}]:} \PY{k}{def} \PY{n+nf}{model}\PY{p}{(}\PY{n}{X\PYZus{}train}\PY{p}{,} \PY{n}{Y\PYZus{}train}\PY{p}{,} \PY{n}{X\PYZus{}test}\PY{p}{,} \PY{n}{Y\PYZus{}test}\PY{p}{,} \PY{n}{num\PYZus{}iterations}\PY{o}{=}\PY{l+m+mi}{2000}\PY{p}{,} \PY{n}{learning\PYZus{}rate}\PY{o}{=}\PY{l+m+mf}{0.5}\PY{p}{,} \PY{n}{print\PYZus{}cost}\PY{o}{=}\PY{k+kc}{False}\PY{p}{)}\PY{p}{:}
             \PY{c+c1}{\PYZsh{} initialize}
             \PY{n}{w}\PY{p}{,}\PY{n}{b}\PY{o}{=}\PY{n}{initialize\PYZus{}with\PYZus{}zeros}\PY{p}{(}\PY{n}{X\PYZus{}train}\PY{o}{.}\PY{n}{shape}\PY{p}{[}\PY{l+m+mi}{0}\PY{p}{]}\PY{p}{)} \PY{c+c1}{\PYZsh{} Gradient descent}
             \PY{c+c1}{\PYZsh{} Retrieve parameters w and b from dictionary \PYZdq{}parameters\PYZdq{}}
             \PY{n}{parameters}\PY{p}{,}\PY{n}{grads}\PY{p}{,}\PY{n}{costs}\PY{o}{=}\PY{n}{backwize}\PY{p}{(}\PY{n}{w}\PY{p}{,}\PY{n}{b}\PY{p}{,}\PY{n}{X\PYZus{}train}\PY{p}{,}\PY{n}{Y\PYZus{}train}\PY{p}{,}\PY{n}{num\PYZus{}iterations}\PY{p}{,} \PY{n}{learning\PYZus{}rate}\PY{p}{,}\PY{n}{print\PYZus{}cost}\PY{p}{)}
             \PY{n}{w}\PY{o}{=}\PY{n}{parameters}\PY{p}{[}\PY{l+s+s2}{\PYZdq{}}\PY{l+s+s2}{w}\PY{l+s+s2}{\PYZdq{}}\PY{p}{]}
             \PY{n}{b}\PY{o}{=}\PY{n}{parameters}\PY{p}{[}\PY{l+s+s2}{\PYZdq{}}\PY{l+s+s2}{b}\PY{l+s+s2}{\PYZdq{}}\PY{p}{]}
             \PY{n}{Y\PYZus{}prediction\PYZus{}train}\PY{o}{=}\PY{n}{predict}\PY{p}{(}\PY{n}{w}\PY{p}{,}\PY{n}{b}\PY{p}{,}\PY{n}{X\PYZus{}train}\PY{p}{)}
             \PY{n}{Y\PYZus{}prediction\PYZus{}test}\PY{o}{=}\PY{n}{predict}\PY{p}{(}\PY{n}{w}\PY{p}{,}\PY{n}{b}\PY{p}{,}\PY{n}{X\PYZus{}test}\PY{p}{)}
             \PY{n+nb}{print}\PY{p}{(}\PY{l+s+s2}{\PYZdq{}}\PY{l+s+s2}{train accuracy: }\PY{l+s+si}{\PYZob{}\PYZcb{}}\PY{l+s+s2}{\PYZpc{}}\PY{l+s+s2}{\PYZdq{}}\PY{o}{.}\PY{n}{format}\PY{p}{(}\PY{l+m+mi}{100}\PY{o}{\PYZhy{}}\PY{n}{np}\PY{o}{.}\PY{n}{mean}\PY{p}{(}\PY{n}{np}\PY{o}{.}\PY{n}{abs}\PY{p}{(}\PY{n}{Y\PYZus{}prediction\PYZus{}train}\PY{o}{\PYZhy{}}\PY{n}{Y\PYZus{}train}\PY{p}{)}\PY{p}{)}\PY{o}{*}\PY{l+m+mi}{100}\PY{p}{)}\PY{p}{)}
             \PY{n+nb}{print}\PY{p}{(}\PY{l+s+s2}{\PYZdq{}}\PY{l+s+s2}{test accuracy:}\PY{l+s+si}{\PYZob{}\PYZcb{}}\PY{l+s+s2}{\PYZpc{}}\PY{l+s+s2}{\PYZdq{}}\PY{o}{.}\PY{n}{format}\PY{p}{(}\PY{l+m+mi}{100}\PY{o}{\PYZhy{}}\PY{n}{np}\PY{o}{.}\PY{n}{mean}\PY{p}{(}\PY{n}{np}\PY{o}{.}\PY{n}{abs}\PY{p}{(}\PY{n}{Y\PYZus{}prediction\PYZus{}test}\PY{o}{\PYZhy{}}\PY{n}{Y\PYZus{}test}\PY{p}{)}\PY{p}{)}\PY{o}{*}\PY{l+m+mi}{100}\PY{p}{)}\PY{p}{)}
             
             \PY{n}{d}\PY{o}{=}\PY{p}{\PYZob{}}\PY{l+s+s2}{\PYZdq{}}\PY{l+s+s2}{costs}\PY{l+s+s2}{\PYZdq{}}\PY{p}{:}\PY{n}{costs}\PY{p}{,}
               \PY{l+s+s2}{\PYZdq{}}\PY{l+s+s2}{Y\PYZus{}prediction\PYZus{}test}\PY{l+s+s2}{\PYZdq{}}\PY{p}{:}\PY{n}{Y\PYZus{}prediction\PYZus{}test}\PY{p}{,}
               \PY{l+s+s2}{\PYZdq{}}\PY{l+s+s2}{Y\PYZus{}prediction\PYZus{}train}\PY{l+s+s2}{\PYZdq{}}\PY{p}{:}\PY{n}{Y\PYZus{}prediction\PYZus{}train}\PY{p}{,}
               \PY{l+s+s2}{\PYZdq{}}\PY{l+s+s2}{w}\PY{l+s+s2}{\PYZdq{}}\PY{p}{:}\PY{n}{w}\PY{p}{,}
               \PY{l+s+s2}{\PYZdq{}}\PY{l+s+s2}{b}\PY{l+s+s2}{\PYZdq{}}\PY{p}{:}\PY{n}{b}\PY{p}{,}
               \PY{l+s+s2}{\PYZdq{}}\PY{l+s+s2}{learning\PYZus{}rate}\PY{l+s+s2}{\PYZdq{}}\PY{p}{:}\PY{n}{learning\PYZus{}rate}\PY{p}{,}
               \PY{l+s+s2}{\PYZdq{}}\PY{l+s+s2}{num\PYZus{}iterations}\PY{l+s+s2}{\PYZdq{}}\PY{p}{:}\PY{n}{num\PYZus{}iterations}\PY{p}{\PYZcb{}}
             \PY{k}{return} \PY{n}{d}
             
\end{Verbatim}


    参考文章:
https://mp.weixin.qq.com/s?\_\_biz=MzI4ODY2NjYzMQ==\&mid=2247484153\&idx=1\&sn=cf3d17f6333855bfd20cbfa3a98bbbf4\&chksm=ec3ba791db4c2e878f72af92a76fcc91a832319da14befbc156bda0cf9c28f688df897c1a541\&scene=21\#wechat\_redirect


    % Add a bibliography block to the postdoc
    
    
    
    \end{document}
